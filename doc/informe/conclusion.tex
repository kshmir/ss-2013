\section*{Conclusión}

La simulación de eventos discretos sirve para encontrar soluciones, a veces
aproximadas, a problemas que no se pueden resolver analíticamente o solamente
con los herramientas de la estadística. Sin embargo, es un método muy formal,
que requiere un gran rigor, para que permita sacar conclusiones y recomendaciones
relevantes. Su otra ventaja es que, además de los resultados, la simulación suele
presentar resultados gráficos que permiten entender mejor el funcionamiento del
sistema, ver donde quedan los cuellos de botella, estudiar las interacciones
que no se pueden predecir en un modelo matemático, etc.

% me falta un poco de chamullo mágico acá a propósito de los resultados

La elección de Ruby on Rails para desarrollar el simulador resultó una 
experimentación muy útil y muy positiva. El lenguaje Ruby, por ser mucho más
lento que Java o C++, no se usa para proyectos de SED muy grandes. 
El mayor problema es la escalabilidad: Ruby no permite, por su construcción,
manejar simulaciones con miles de entidades a la vez y grandes colecciones de datos.
Sin embargo, para estudios sobre sistemas bastante pequeños y que no requieren 
una cantidad increíbles de corridas, la facilidad de programación de Ruby 
es una ventaja enorme. 

