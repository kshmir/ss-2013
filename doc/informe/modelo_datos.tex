\section{Modelo de datos}

A los dos tipos de eventos que contemplamos en nuestro modelo corresponden dos muestras artificiales.
La primera variable aleatoria sirve para definir los tiempos entre llegadas, la segunda para los 
tiempos de procesamiento de los pedidos en los servidores. Para definir las distribuciones de esas
variables aleatorias, nos basamos en el artículo de RapGenius. Ellos estudiaron las distribuciones
de llegadas y de salidas en su propio caso y utilizar los mismos datos nos permite comparar nuestros
resultados con los suyos.

\paragraph{Distribución de tiempo entre llegadas}
Los pedidos llegan al \textit{router} según un proceso de Poisson. Según RapGenius, hay un promedio
de nueve mil pedidos por minutos. Elegimos como unidad de tiempo en todo el modelo la milisegundo;
entonces nuestra primera muestra artificial usa una distribución de Poisson de parámetro 
$\lambda = 6.667 \mathrm{ms}$.

\paragraph{Distribución de tiempo de procesamiento}
En el ámbito de aplicaciones web, la amplitud del intervalo de los tiempos de procesamiento 
es muy grande. 
Según RapGenius, el tiempo mínimo es de $10\mathrm{ms}$ y el máximo de $30000\mathrm{ms}$.
La distribución acumulada de los tiempos es la siguiente:
\begin{table}
    \centering
    \begin{tabulary}{\textwidth}{|C|C|C|C|C|C|C|C|C|}
        \hline
        1\% & 5\% & 10\% & 25\% & 50\% & 75\% & 90\% & 99\% & 99.9\% \\
        \hline
        7ms & 8ms & 13ms & 23ms & 46ms & 255ms & 923ms & 3144ms & 7962ms \\
        \hline
    \end{tabulary}
    \caption{Distribución de los tiempos de procesamiento}
    \label{tab:distro-procesamiento}
\end{table}
Esta distribución se aproxima muy bien con $Weibull(shape=0.46,\;scale=111)$ modificada
para tener en cuenta un mínimo de $10\mathrm{ms}$ y un máximo de $30000\mathrm{ms}$.
