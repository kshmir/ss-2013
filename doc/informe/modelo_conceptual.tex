\section{Modelo conceptual}

\subsection{Simplificaciones}

Nuestro modelo es una maqueta que simula el funcionamiento lógico de un verdadero balanceador de carga. Sin embargo, los
balanceadores de carga, por el papel que tienen en el funcionamiento de la red, tienen una complejidad mucha mas
importante que la de nuestro modelo. En nuestro simulador, todo el funcionamiento de la red informática es una
abstracción. El \textit{router} solamente se encarga de decidir a que servidor mandar pedidos que le llegan. No
contemplamos tampoco las fallas y los errores lógicas y materiales que pueden ocurrir en un verdadero sistema.

\subsection{Diagrama de bloques}

Nuestro sistema funciona de la manera presentada en la figura \ref{fig:diagrama-bloques}. Cuando la simulación empieza,
no hay pedidos pendientes en el sistema, y el primer pedido está llegando. Cada etapa de la simulación corresponde a un
evento, o sea un arribo de un pedido al \textit{router}, o una salida de un pedido de un servidor. En el diagrama, se
nota que cada pedido pasa por la cola del \textit{router} antes de llegar a un servidor. Sin embargo, en nuestro modelo,
hacemos la simplificación que todo el tiempo que el pedido pasa en el sistema se divide entre la espera en colas y el
procesamiento en un servidor. Los tiempos de tránsito o de \textit{routing} están considerados como despreciables.
Entonces, si es posible mandar directamente el pedido a un servidor cuando entra en la cola del \textit{router}, el
tiempo de principio de procesamiento en el servidor es igual al tiempo de llegada al \textit{router}, o sea el paso por
la cola principal no afecta el pedido en este caso.

\begin{figure}[h]
    \centering
    \includegraphics[width=\textwidth]{diagrama_bloques.png}
    \caption{Diagrama de bloques}
    \label{fig:diagrama-bloques}
\end{figure}

En la figura \ref{fig:diagrama-bloques}, el bloque más importante para nuestro estudio es el bloque en verde. La
elección del servidor puede hacerse con cualquier de los cuatro algoritmos de \textit{routing}.


\subsection{Plan de cuadros}

\subsubsection{Valores relevantes del modelo}

\paragraph{Variable de control}
En nuestro estudio, tenemos una sola variables de control: el algoritmo utilizado para elegir un servidor al que mandar
un pedido. No es una variable cuantitativa así que no se espera encontrar el mejor algoritmo en absoluto sino observar
las ventajas y desventajas de cada algoritmo de forma objetiva.

\paragraph{Funciones objetivos}
Contemplamos varios resultados:
\begin{itemize}
    \item el tiempo promedio que un pedido pasa en el sistema, entre el momento en que llega al \textit{router} hasta el
	momento en que sale del servidor que lo atendió
    \item el tiempo ocioso promedio de un servidor
    \item el número de pedidos rechazados
\end{itemize}
De la segunda función, se estudia también el desvío entre servidores. Una calidad que se busca en un balanceador de
carga es la equidad: es decir, todos los servidores deberían tener una carga parecida.


\paragraph{Parámetros}
Algunos valores pueden afectar la simulación. Eses parámetros son los siguientes:
\begin{itemize}
    \item el número de servidores
    \item el tamaño máximo de la cola de \textit{router}
    \item el tamaño máximo de la cola de cada servidor, en el caso de \textit{Smart routing}, no hay colas en los servidores
    \item la cantidad de corridas
\end{itemize}


\subsubsection{Resultados esperados}

\begin{table}
    \centering
    \begin{tabulary}{\textwidth}{|c||C||C|C|C|}
    \hline
    Algoritmo & Número de servidores & Duración promedia de un pedido & Tiempo ocioso promedio & Desvío entre servidores del tiempo ocioso \\
    \hline
	\multirow{4}{1.8cm}{Random routing} & 30 &  &  &  \\
                                        & 40 &  &  &  \\
                                        & 50 &  &  &  \\
                                        & 60 &  &  &  \\
	\hline
    \multirow{4}{1.8cm}{Round-robin routing} & 30 &  &  &  \\
                                             & 40 &  &  &  \\
                                             & 50 &  &  &  \\
                                             & 60 &  &  &  \\
	\hline
    \multirow{4}{1.8cm}{Shortest queue routing} & 30 &  &  &  \\
                                                & 40 &  &  &  \\
                                                & 50 &  &  &  \\
                                                & 60 &  &  &  \\
	\hline
    \multirow{4}{1.8cm}{Smart routing} & 30 &  &  &  \\
                                       & 40 &  &  &  \\
                                       & 50 &  &  &  \\
                                       & 60 &  &  &  \\
	\hline
    \end{tabulary}
    \caption{Cuadro comprativo de los resultados esperados}
    \label{tab:plan-cuadros}
\end{table}



Los resultados que esperamos están presentados en el cuadro \ref{tab:comparativo-esperados}. Este cuadro se lee de la 
siguiente manera: para ciertos criterios, que consideramos relevantes, hicimos una clasificación de los algoritmos. Para
cada criterio, el algoritmo clasificado como ``1'' es el mejor, y el ``4'' el peor. 

\begin{table}
    \centering
    \begin{tabulary}{\textwidth}{|c||C|C|C|}
    \hline
    Algoritmo & Complejidad  de implementación & Equidad en la ocupación & Duración promedia de un pedido \\
    \hline
	Random routing & 2 & 4 & 4\\
	\hline
	Round-robin routing & 1 & 3 & 3\\
	\hline
	Shortest queue routing & 4 & 1 & 2\\
	\hline
	Smart routing & 3 & 2 & 1\\
	\hline
    \end{tabulary}
    \caption{Cuadro comprativo de los resultados esperados}
    \label{tab:comparativo-esperados}
\end{table}

\subsection{Diagrama de flujo}

%falta el diagrama de flujo por acá
Tomamos la decisión de incluir en el diagrama de flujo únicamente la lógica del modelo,
olvidándonos a propósito de las ecuaciones y de los pasos que describen la colección de los
datos estadísticos. La motivación principal para eso es aliviar lo más posible el diagrama
para hacer hincapié en la lógica. Además, si bien un diagrama de flujo describe con exactitud
la lógica de un algoritmo, no sirve para diseñar el almacenamiento de los datos, y esta parte
tiene demasiada importancia a la hora de describir la colección de datos estadísticos para
ignorarla.


