\section{Modelo operacional}

\subsection{Simulador}

Nuestro simulador es una aplicación web basada en el \textit{framework} \emph{Ruby on Rails}.
Consta de dos partes: el núcleo que se encarga de correr la simulación y colectar los resultados
y la página web que presenta los resultados y la animación de la simulación.
El núcleo está programado en \emph{Ruby} y la animación de la página web en \emph{JavaScript}.


El simulador sigue exactamente la lógica del diagrama de flujo pero viene además con funciones
para la colección de datos estadísticos. Ruby es un lenguaje orientado a objetos y de muy alto
nivel, lo que nos permitió centrarnos en la lógica del algoritmo. El núcleo se comunica con la
parte de presentación a través de un documento en el formato ``json''. Este formato es muy
simple y permite el intercambio de datos de proceso a proceso de manera muy eficiente.


% acá, podemos describir un poco la UI

\subsection{Plan de experimentación}

Nuestro plan de experimentación sigue basicamente el plan de cuadros

