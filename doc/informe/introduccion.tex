\section{Introducción}

El uso de la simulación de eventos discretos permite resolver problemas que tienen una complejidad que impide alcanzar
a una solución con la análisis o la teoría de las probabilidades. Los casos típicos son problemas de ingeniería de la
producción donde las interacciones entre entidades (que pueden ser humanos, productos, máquinas, computadoras, etc.) y
la cantidad de objetos que componen el sistema completo en cualquier momento son tan complejas que resultaría imposible
manejar un modelo matemático correspondiente.

En nuestro caso, el sistema real que vamos a estudiar es un sistema de balanceo de carga en un ámbito de procesamiento
de pedidos web. Los dispositivos de balanceo de carga son \textit{routers} que reciben pedidos y los distribuyen a
servidores capaces de atenderlos. En el caso que nos interesa estudiar, consideramos servidores alojando aplicaciones
web a las que usuarios quieren acceder. Cada servidor puede atender uno solo pedido a la vez. Entonces, para permitir a
la aplicación de ser concurrente, y escalable, se usan varios servidores corriendo la misma aplicación. Del punto de
vista del usuario externo, eso no cambia nada porque accede a la aplicación conociendo la dirección de la página web.
Pero, internamente, el \textit{router} manda su pedido a determinado servidor. Así, la cantidad de servidores corriendo
la aplicación es directamente vinculado a la ``concurrencia'' de la aplicación, es decir la cantidad de clientes que
pueden ser atendidos a la vez.

Sin embargo, esa concurrencia no es necesariamente \emph{proporcional} a la cantidad de servidores. Como lo muestra la
formula de Little, la cantidad de entidades en un sistema no es igual a la cantidad de entidades en las colas más el
número de puestos de atención, sino a la cantidad de entidades en las colas más \emph{la tasa de tránsito}, es decir el
cociente entre el tiempo promedio entre dos arribos y el tiempo promedio de atención de una entidad. Además, en nuestro
caso, la mejora de la concurrencia según el aumento del número de servidores depende también de la política de
\textit{routing} elegida.
